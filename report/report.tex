\documentclass[a4paper]{article}
\usepackage[utf8]{inputenc}
\usepackage{float}
\usepackage{pdfpages}
\usepackage{textcomp}
\usepackage{indentfirst}

\setlength{\parskip}{1em}

% Colour Management
\usepackage{color}

% Multi-Line Comments
\usepackage{comment}

% Customisable Sections
\usepackage{titlesec}

% Images and Captions
\usepackage{graphicx}
\usepackage{subcaption}
\usepackage{wrapfig}
\graphicspath{{./images/}}
\usepackage{fancyhdr} 


% Bibliography
\usepackage[nottoc]{tocbibind}
\usepackage[
    backend=biber,
    style=ieee]{biblatex}

\addbibresource{report_citations.bib} %Imports bibliography file



% Tables
\usepackage{xcolor}
\usepackage{array}
\newcolumntype{L}{>{\raggedright\arraybackslash}m{0.9\linewidth}}

% Contents Page
\usepackage{hyperref}

% Appendices
\usepackage[toc,page]{appendix}

% Bold Maths Symbols
\usepackage{bm}

% Allowing lower levels of Contents Page
\setcounter{tocdepth}{4}
\setcounter{secnumdepth}{5}






\title{Engineering Design Project}
\author{Georgio Chaimali \and Dimitrios Georgakopoulos \and Edvard J.S. Holen 
        \and Hyunjoon Jeon \and Josiah Mendes \and Raghav Viswakumar}
\begin{document} 
\begin{titlepage}
    \setlength{\headheight}{66.89pt}
    \thispagestyle{fancy}
    \renewcommand{\headrulewidth}{0pt}
    \renewcommand{\footrulewidth}{0pt}
    \lhead{\includegraphics[scale=0.1]{logo.png}}
    \cfoot{} % this is to remove the page number
    \hbox{}\vfill
    \begin{center} 
	    {\scshape\LARGE Imperial College London  \par}
	    \vspace{1cm}
        {\scshape\Large Second Year Design Project\par}
        \vspace{0.25cm}
        {\scshape\Large ELEC50003/ELEC50008\par}
        \vspace{1.5cm}
        {\huge\bfseries The MARS Rover\par}
        \vspace{2cm}
        {\Large\itshape Gorup 1\par}
        \vfill
        \begin{flushright}
            \textsl{ \large
            Georgio Chaimali \\ Dimitrios Georgakopoulos \\ Edvard J.S. Holen 
            \\ Hyunjoon Jeon \\ Josiah Mendes \\ Raghav Viswakumar
            }
        \end{flushright}
        \vfill

        % Bottom of the page
        {\large Word Count: XXXX Words \\ \today\par}
    \end{center}
\end{titlepage}
 

\tableofcontents

\newpage

\section{Overview}

\section{Systems}

\begin{figure}[H]
\centering
\includegraphics[scale=0.18]{logo.png}
\caption{Sample Figure}
\label{fig:image1}
\end{figure}

Sample Reference\cite{einstein}

\subsection{Control}
\subsection{Comms}
\subsection{Energy}
Hello

\includegraphics[scale=0.3]{Series(S)}
\subsection{Vision}
\begin{abstract}
    The purpose of the Vision module is threefold:
    1. Capture data from camera module;
    2. Detect objects of interest within the current view and send their location to the Control module; and
    3. Send image data to Control for streaming to Command. 
\end{abstract}

\subsubsection{Hardware Organisation}

The Vision module comprises of two main hardware elements: 
    the Terasic DE10-Lite, a cost-effective Altera MAX 10 based FPGA board \cite{TerasicDE10Web} 
    and the Terasic D8M-GPIO camera package \cite{TerasicD8MWeb}
that interfaces with the FPGA through the onboard GPIO connectors. 

These hardware choices were made by the project organisers, 
but are also sufficient and capable of carrying out the tasks at hand. 
As the FPGA's hardware is configurable, 
it is more flexible than other embedded systems that are limited to a general purpose processor,
and is also able to handle both streaming and processing of high resolution images
without significant compromises on framerate or data speed 
through the use of concurrent operations and dedicated blocks for signal processing applications like multiplication.
This particular FPGA is also equipped with a 4-bit VGA output which is useful for debugging object detection live, 
and also has a connector for an Arduino Uno R3 shield, \cite{TerasicDE10Web} 
which can be used to interface with the ESP32 used for control.  

In order to perform general purpose operations like
    to configure camera settings
    and to provide a debugging interface,
a Nios II soft core was instantiated on the FPGA. 
Alternatively, to implement a more advanced image processing algorithm
or to reduce hardware components, 
a FPGA with a hard core, known as a FPGA System-On-Chip (FPGA SoC) \cite{FPGASoC}




\section{Evaluation and Conclusion}

\newpage

\printbibliography[
heading=bibintoc,
title={References}
]

\end{document}