\documentclass[a4paper]{article}
\usepackage[utf8]{inputenc}
\usepackage{float}
\usepackage{pdfpages}
\usepackage{textcomp}
\usepackage{indentfirst}

\setlength{\parskip}{1em}

% Colour Management
\usepackage{color}

% Multi-Line Comments
\usepackage{comment}

% Customisable Sections
\usepackage{titlesec}

% Images and Captions
\usepackage{graphicx}
\usepackage{subcaption}
\usepackage{wrapfig}
\graphicspath{{./images/}}
\usepackage{fancyhdr} 


% Bibliography
\usepackage[nottoc]{tocbibind}
\usepackage[
    backend=biber,
    style=ieee]{biblatex}

\addbibresource{report_citations.bib} %Imports bibliography file


% Tables
\usepackage{xcolor}
\usepackage{array}
\newcolumntype{L}{>{\raggedright\arraybackslash}m{0.9\linewidth}}

% Contents Page
\usepackage{hyperref}

% Appendices
\usepackage[toc,page]{appendix}

% Bold Maths Symbols
\usepackage{bm}
\usepackage{amsmath} %sudo tlmgr install amsmath
\usepackage{gensymb} %sudo tlmgr install was
\usepackage{multicol}



% Allowing lower levels of Contents Page
\setcounter{tocdepth}{4}
\setcounter{secnumdepth}{5}

% Make margins smaller, feel free to change
\usepackage{geometry}
\geometry{
    a4paper,
    top = 20mm,
    bottom = 20mm,
    textwidth=426pt
}





\title{Engineering Design Project}
\author{Georgio Chaimali \and 
        Dimitrios Georgakopoulos \and 
        Edvard J. Skaarberg Holen \and 
        Hyunjoon Jeon \and 
        Josiah Mendes \and 
        Raghav Viswakumar}
\begin{document} 
\begin{titlepage}
    \setlength{\headheight}{66.89pt}
    \thispagestyle{fancy}
    \renewcommand{\headrulewidth}{0pt}
    \renewcommand{\footrulewidth}{0pt}
    \lhead{\includegraphics[scale=0.1]{logo.png}}
    \cfoot{} % this is to remove the page number
    \hbox{}\vfill
    \begin{center} 
	    {\scshape\LARGE Imperial College London  \par}
	    \vspace{1cm}
        {\scshape\Large Second Year Design Project\par}
        \vspace{0.25cm}
        {\scshape\Large ELEC50003/ELEC50008\par}
        \vspace{1.5cm}
        {\huge\bfseries The MARS Rover\par}
        \vspace{2cm}
        {\Large\itshape Group 1\par}
        \vfill
        \begin{flushright}
            \textsl{ \large
            Georgio Chaimali \\ 
            Dimitrios Georgakopoulos \\ 
            Edvard J. Skaarberg Holen \\ 
            Hyunjoon Jeon \\ 
            Josiah Mendes \\ 
            Raghav Viswakumar
            }
        \end{flushright}
        \vfill

        % Bottom of the page
        {\large Word Count: XXXX Words \\ \today\par}
    \end{center}
\end{titlepage}
 
\setcounter{tocdepth}{2}
\tableofcontents

\newpage

\section{Overview}

\section{Systems}

\subsection{Control}

\subsubsection{Overview of Module and Goals}
The Control module within the context of the rover has one main goal: 
to act as the communication hub between all the subsystems, delivering
the relevant data where it is required to allow for the rover to integrate
as a whole \cite{MarsRoverSpec}. There are a couple of core objectives which must be 
achieved in order to fulfill this role and they are as follows:

\begin{enumerate}
    \item Act as a Wifi Access Point sitting under a router to receive
    commands from the Command subsystem and send data through socket-level communication.
    \item Use a relevant hardware-level data transmission protocol to
    send movement commands to the Drive subsystem and receive data for Command. 
    \item Use a relevant hardware-level data transmission protocol to receive
    the Vision subsystem data.
    \item Connect the Energy subsystem with the rover, sending the relevant data
    to the Command subsystem.
\end{enumerate}

Although the hardware is fixed, the solutions to the outlined problems
above are mostly up to design. It is for this reason that the main challenge
stems from understanding what works best for communication to each subsystem
and more importantly, how to integrate it such that the transmission of data
can be smoothly facilitated through the ESP32.

\subsubsection{Hardware Organisation}

The Control module comprises of two hardware components \cite{BoxContent}: 
the ESP32, a system on chip microcontroller \cite{ESP32Datasheet}, as well as an Arduino adapter
board designed to map some of the available GPIO pins on the ESP32 to an Arduino-like
board \cite{ESP32ArduinoAdapter}.

This hardware was provided by the project organisers and it immediately limits
the tools which can be used, however the wide array of functionality which
this microcontroller can provide through programming the chip allows
for sufficient freedom and capabilities to complete the required tasks. The
GPIO pins are highly configurable when compared to an Arduino-Uno \cite{MicrocontrollerComparison}, as they
are capable of using more data communication through almost any of the pins
via software configuration \cite{ESP32PinOut}. It is low-cost and low-power, 
capable of using Wifi capabilities at 2.4GHz and Bluetooth \cite{ESP32Datasheet} to interface wirelessly
with other devices. Using the Arduino Adapter board \cite{ESP32ArduinoAdapter}, it can sit directly on top of
an Arduino or device with similar pins to directly interface through physical connections. 


\subsubsection{Drive and Control}

\subsubsection{Vision and Control}

\subsubsection{Command and Control}

\subsection{Comms}

\subsection{Vision}
\begin{abstract}
    The purpose of the Vision module is threefold:
    1. Capture data from camera module;
    2. Detect objects of interest within the current view and 
        send their location to the Control module; and
    3. Send image data to Control for streaming to Command. 
\end{abstract}

\subsubsection{Hardware Organisation}

The Vision module comprises of two main hardware elements: 
    the Terasic DE10-Lite, a cost-effective Intel MAX 10 based FPGA board 
    \cite{TerasicDE10Web} 
    and the Terasic D8M-GPIO camera package \cite{TerasicD8MWeb}
that interfaces with the FPGA through the onboard GPIO connectors. 

These hardware choices were made by the project organisers, 
but are also sufficient and capable of carrying out the tasks at hand. 
As the FPGA's hardware is configurable, 
it is more flexible than other embedded systems 
that are limited to a general purpose processor,and 
is also able to handle both streaming and processing of high resolution images
without significant compromises on framerate or data speed 
through the use of concurrent operations and dedicated blocks 
for signal processing applications like multiplication.
This particular FPGA is also equipped with a 4-bit VGA output 
which is useful for debugging object detection live, 
and also has a connector for an Arduino Uno R3 shield, \cite{TerasicDE10Web} 
which can be used to interface with the ESP32 used for control.  

In order to perform general purpose operations like
    to configure camera settings
    and to provide a debugging interface,
a Nios II soft core was instantiated on the FPGA. 
Alternatively, to implement more advanced image processing algorithms
or to reduce other hardware components in the system like the multiple Arduinos, 
a FPGA with a hard core connected via PCIe, 
known as a FPGA System-On-Chip (FPGA SoC) \cite{FPGASoC} could be used, 
which would provide both the advantages of having reconfigurable hardware,
a more capable general purpose processor and a higher bandwidth limit. 

\subsubsection{Image Capture \& Processing Stream}

The image capture and buffering is based on a starter project provided
by Terasic Inc for the D8M Camera module that was modified by Dr Edward Stott 
\cite{EEE2Rover} and provided to us for this project. The system makes use of 
several IP components from the Intel Video and Image Processing Suite,
namely the Clocked Video Output and Frame Buffer Core. The system is compromised
of several modules, that take the raw data from the camera in Bayer filter form\cite{TerasicD8MWeb},
transform it into RGB video packets, buffer the frames to allow camera and output
to run at different frame rates, process the image data and convert the video 
data to serial data for output over VGA.\cite{EEE2Rover} As this system and its 
corresponding documentation is openly available, the implementation outside of 
the image processor will not be described in more depth, interested readers can 
consult the provided repository and starter project for reference. 

\subsubsection{Design \& Implementation of Image Processing Module}

The challenge for this rover was to detect different coloured ping-pong balls as
obstacles and relay the information about these obstacles to the control module. 
The provided ping-pong balls came in 5 different colours - pink, orange, light 
blue, grey and green. As these colours cover a wide range of the colour spectrum
and even overlap with each other (orange and pink, blue and grey), it was 
necessary to implement an algorithm that could take into account the location 
and shape of the ball, as a purely colour based algorithm would be unable to 
differentiate other objects in view of the camera and confuse them with the ball, 
for example, the sky and the light blue ball would be seen as one object. 
 
%Maybe add Image of all the balls here. 


The initial design and implementation of this image processing algorithm involved
a transform from the RGB (Red, Green, Blue) colour space to the HSV (Hue, 
Saturation, Value) colour space, and pixel based classification to detect large
areas of the necessary colours.  HSV has been shown to be a better colour space
for image processing tasks like colour segmentation in road sign detection for 
autonomous driving as it is invariant to illumination changes unlike the RGB 
space.\cite{ali2013performance} As the task at hand here is similar, a transformation
from the RGB space to the HSV space is applied. 

HSV is traditionally expressed as three values ranging between 0-360\degree\  
for hue, and two values between 0 and 1 for saturation and value.\cite{10.1145/965139.807361}
As floating point calculations in SystemVerilog are computationally expensive, 
and also introduce a layer of unnecessary complexity, the conversion was adjusted
to express all three values in terms of an 8 bit integer ranging from 0-255. The
algorithm for conversion is presented mathematically:

\begin{multicols}{3}
    \noindent
    \begin{align*}
        M &= \max(R, G, B) \\ \\
        m &= \min(R, G, B) \\ \\
        C &= M-m 
    \end{align*}
    \begin{equation*}
        H = \begin{cases}
            M = R, & 0 + \frac{43 \times (G-B)}{C} \\
            M = G, & 85 + \frac{43 \times (B-R)}{C} \\
            M = B, & 171 + \frac{43 \times (R-G)}{C} \\
            M = 0, & 0 \\
            C = 0, & 0
        \end{cases} 
    \end{equation*}
    \begin{align*}
        S &= \begin{cases}
            M = 0, & 0 \\ C = 0, & 0 \\ \text{else} & \frac{255\times C}{V}
        \end{cases} & \\ \\
         V &= M  
    \end{align*}
\end{multicols}



Although the inclusion of a divide operation was undesirable as it does not 
translate well to hardware implementations, it was unavoidable with hue being 
obtained by dividing the difference of two components by the chroma value. This 
module was then implemented using two pipelined 3 cycle divider from the Intel IP
megafunction library for the hue and saturation value calculations.  



\subsection{Drive}

% Start of energy subsection
\subsection{Energy}
\begin{abstract}
The main goal of the energy sub-module is to design a battery pack for the rover and charge it using solar power. With this goal in mind, the energy sub-module must develop a battery management system which allows the tracking of battery SOC and SOH, and if necessary perform SOH maintenance. 
    
\end{abstract}

\subsubsection{Characterising Components}
When designing a system it is necessary to know the behaviour and limitations of its constituent components. There are three main components that make up the energy subsystem: the battery cells, the PV panels and the SMPS.

\textbf{\chapter{Battery Cells}} 
\newline
\newline
To determine the behaviour of the battery cells they were all tracked through a full charge/discharge 
cycle using the provided “Battery\_Charge\_Cycle\_Logged\_V1.1.ino” code\cite{chargeCode}. 
Every cell behaved similarly in terms of the cell voltage compared to time. 
The cell voltage of cell 1 over a full discharge/charge cycle is shown in Figure~\ref{fig:charge_cycle}:

\begin{figure}[H]
    \centering
    \includegraphics[scale=0.18]{charge_cycle.png}
    \caption{The voltage evolution of cell 1 through a full discharge/charge cycle.}
    \label{fig:charge_cycle}
    \end{figure}

Note the following important points on the graph. At 200 s the cell is done charging and
enters an idle state for 30 s after which it starts discharging. At ~8000 s the cell is fully
discharges and enters an idle state for 30s after which it starts charging. Finally, at 18800 s the cell is once 
again fully charged and the charge cycle is completed.

The provided charging algorithm also logs the current into the cell.
By integrating said current for a full charge or discharge section
we can determine the cell capacity in mAh. The results of this analysis is 
presented in the table below:

\begin{center}
    \begin{tabular}{||c| c c c c c||} 
    \hline
   Cell Number& 1 & 2 & 3 & 4 & 5 \\ [0.5ex] 
    \hline
    Capacity (mAh) & 542.7	& 526.1	& 519.5	& 530.1	& 543.7\\ [1ex] 
    \hline
   \end{tabular}
   \end{center}


As expected all cells have a capacity somewhere around 500 mAh. However, some cells
are have a higher capacity than others which may have implications for the performance
of certain battery cell configurations.

\textbf{\chapter{PV panels}} 
\newline
\newline
The provided PV panels are rated for a maximum power of 1.15 W at a voltage 
of 5.0 V and current 230 mA. Away from the maximum power point the performance of 
the panels can be determined from their I-V curves. To find the I-V curves each 
panel was connected to the B-inputs of the SMPS operating in non-synchronous boost.
They were then lit by the lamp and the duty cycle of the SMPS was varied while measurements 
of panel current and voltage were taken. After processing the resulting data
is plotted in Figure~\ref{fig:IV_curve}.

 \begin{figure}[H]
    \centering
    \includegraphics[scale=0.18]{Panel1.png}
    \includegraphics[scale=0.18]{Panel2.png}
    \includegraphics[scale=0.18]{Panel3.png}
    \includegraphics[scale=0.18]{Panel4.png}
    \caption{I-V curves for the PV panels.}
    \label{fig:IV_curve}
    \end{figure}

Though the data is noisy, it is clear that all panels exhibit the standard 
I-V characteristics of a PV cell. That is, they behave as non-ideal current 
sources with a nearly constant current at low voltages and a rapid current 
reduction at high voltages\cite{green}. Moreover, we see that the provided lamp activates 
the panels poorly as the peak power for each of the panels is only ~0.5 W.

\textbf{\chapter{SMPS}} 
\newline
\begin{wrapfigure}{r}{0.5\textwidth}
    \begin{center}
      \includegraphics[width=0.48\textwidth]{Boost_efficiency_wduty.png}
    \end{center}
    \caption{SMPS efficiency versus output current with input voltage 5V}
    \label{fig:efficiency}
  \end{wrapfigure}
The provided SMPS is rated for 10 W throughput with a maximum boost output voltage of 35 V 
and maximum output current of 10 A\cite{SMPS_lab}. All these ratings are far higher 
than needed and neither is expected to impose limitations on the design of the energy module. 

The many characteristics of the SMPS have been thoroughly examined in 
2nd year labs. However, for the energy submodule the most important characteristics 
will be the SMPS efficiency during non-synchronous boost operation. A graph of 
efficiency versus output current is shown in Figure~\ref{fig:efficiency}.


\subsubsection{Battery Configuration}




When designing the battery pack there are two principal choices that need to be made. 
Firstly, how many battery cells should the battery pack consist of and, secondly, 
in what manner should these cells be connected? 

The optimal number of cells is in large part set by the energy 
and power needs of other submodules. During testing the total power 
draw of the rover was found to be about 2 W. Each battery cell has a 
nominal voltage of 3.2 V and a maximum peak discharge current of 1 A 
[battery datasheet], giving a maximum power draw of 3.2 W per cell. 
Thus, to meet the power requirements of the rover it is enough to use only 
one cell. However, each cell can only store about 3.2 V * 0.5 A * 3600 s = 5760 J, 
which would only power the rover for 48 minutes. It is desirable to have 
the rover be operational for as many hours as possible each day. 
Assuming that 12 hours a day are completely without sunlight it is clear 
that the rover cannot work through the night even if all 5 available 
battery cells were used. To give the rover the most operational hours we 
would then want to use all 5 battery cells. However, connecting 5 battery 
boards to the Arduino would use at least 11 of the 12 free Arduino pins, 
leaving only one pin for all other purposes. As such it might be wise to use 
less cells. To accommodate other circuit functions, only 4 cells will therefore be used.

Now consider the PV array. Each PV panel is rated for 1.15 W, meaning 
that the array as a whole should produce 4.6 W. As shown in figure ? the 
peak efficiency of the SMPS in boost mode is about 80\% giving a maximum 
usable power of 0.80*4.6 W = 3.7 W. Assuming 12 hours of sunlight in a day, 
this means that the PV array will produce less energy each day than the rover 
uses in 24 hours. It is therefore of high priority to capture as much of the 
solar power as possible. The battery cells have a standard charging current of 
250 mA [battery datasheet]. The power needed to charge the at this current is 
shown in the table below:

\begin{center}
    \begin{tabular}{||c| c c c c c||} 
    \hline
    Number of cells& 1 & 2 & 3 & 4 & 5 \\ [0.5ex] 
    \hline
    Nominal charging power (W) & 0.8 & 1.6 & 2.4 & 3.2 & 4.0\\ [1ex] 
    \hline
    Peak charging power (W) & 0.9 & 1.8 & 2.7 & 3.6 & 4.5 \\ [1ex] 
    \hline
   \end{tabular}
   \end{center}


A 4 cell battery pack is the largest battery pack which has a power 
requirement of less than 3.7 W at standard charging current and little 
power goes unused. Adding to this that the rover has most operational 
hours with 4 cells, using 4 cells is the clear choice. Note that the 
cells used are number 1, 2, 4 and 5 as these were found to have the 
highest capacity.

There are several ways to connect the 4 cells into a power pack, 
however the design brief advised against mixing parallel and series 
connections [phils video]. As such only fully series and parallel 
battery packs will be considered. 

A series battery pack has some obvious disadvantages compared to 
a parallel battery pack. In section 2.5.1 it was shown that all the 
battery cells have different capacities. In a parallel battery pack 
this is not a big problem, as it is possible to draw different currents 
from each cell and thereby use the full capacity of each cell. For a 
series battery pack however, the total battery capacity will be limited 
by the cell with the lowest capacity. As such, a series battery pack 
is able to store less usable energy than an equivalent parallel battery 
pack. Moreover, to check the OCV of each cell they need to be switched 
out of circuit using the battery board relay. For a series battery pack 
this leads to an open circuit and any charging/discharging of the battery 
must halt while the voltage is measured. For a parallel battery pack on the 
other hand, switching the relay of a cell only takes that one cell out of 
circuit and it is possible to charge/discharge the battery pack while 
taking voltage measurements. Being able to switch single cells out of 
circuit can also enables switching faulty cells out of circuit, meaning 
that one cell failing would not mean that the battery pack as a whole fails. 
Finally, a parallel battery pack is self-balancing and energy automatically
 flows between cells when necessary [needs reference]. Series cells on the 
 other hand need to be balanced by switching on and off balancing resistor. 
 This not only is more complex, but leads to power being lost in the balancing
  resistors.

There is however one major weakness to a parallel battery pack: 
it is very hard to track the current going into individual cells. For 
safe operation it is necessary to prevent over-current into each individual 
cell. However, the current sensor on the SMPS can only measure the current 
flowing into the battery pack as a whole. Each battery cell is only rated 
for a rapid charging current of 500 mA [battery cell datasheet],  and seeing 
as there is no way to know how the current splits into each of the cells 
one must operate one must operate with the assumption that all the current 
can flow into a single cell. As such, no more than 500 mA can be allowed 
to flow into the battery pack as a whole. At this current the nominal 
charge power is only 3.2V*0.5W = 1.6 W meaning charging will be slow and 
it is only possible to use less than half of the available solar power. 
This is such a large drawback of using a parallel battery pack, that 
despite all the previously stated disadvantages of a series battery pack, 
a series battery pack has been deemed the best option.

\subsubsection{PV Array Configuration}
    \includegraphics[scale=0.3]{Series(S)}

\subsubsection{SMPS Configuration}

\subsubsection{Maximum Power Point Tracking}

\subsubsection{Charging Algorithm}

\subsubsection{Discharging Algorithm}

\subsubsection{Safety Mechanisms}

\subsubsection{State of Charge}

\subsubsection{State of Health}
maintenance

\subsubsection{Communicating with Other Modules}
Though it is not necessary to fully integrate the energy module with the rest of the rover, other submodules, specifically command, needs access data such as the battery SOH and SOC. For communicating with other modules the Arduino shield has a set of UART ports. However, as group members were not in the same location it was not possible to physically connect the energy module to the rover, which is necessary to use UART. As such, an alternative approach was employed. First the Arduino was connected to a computer via USB. On the computer a Python script was run [8]. At the start the Python script establishes a connection to a server created by running a similar script on the command module [9]. After a connection has been established the Python script starts reading the serial data coming from the Arduino and transmits it using TCP to the command module. Each message coming from the Arduino is in CSV form where the first entry is the message ID, which allows the command script to decode what type of data is being sent. 

\subsubsection{Physical Integration of the Energy Module}

% End of energy subsection

\subsection{Integration}



\section{Evaluation and Conclusion}

\section{Project Management and Organisation}

As this project was carried out remotely
with contributors located in different countries, 
it was important to have a good framework for communication and management. 

The main tool used for communications and management was Git + GitHub.
As the codebase was incredibly complex, 
involving many different libraries and with each submodule being capable as a
standalone project, it was vital to have a version control system in place. 
Being able to keep a history of commits and changes made to a project was useful,
especially when trying to track down the origin of a bug and what caused it. 

The team also made use of GitHub Issues to track progress and accountability in 
the initial design phase. A thread was opened for each submodule to show what 
the lead for that submodule had been doing and potential avenues of achieving 
their goals. This was beneficial both for the leads to keep track of their 
research, but also allowed other members to contribute to other submodules 
by adding comments and voicing their thoughts. GitHub Issues were also linked 
directly to commits in the codebase to allow for a more in-depth explanation and
reasoning with context for a commit than what is allowed in the commit message 
area. 

Simultaneously, a Gantt chart was maintained to keep track of progress and is 
available for viewing under the maintained GitHub repository linked in the Appendix.

%TODO #7!
\section{Intellectual Property}
\begin{comment}
Could link stuff about how we are using Intel IP in the FPGA, as well as things
like licencing and how some companies make their entire business based on it,
like Arm, hence ethically it is important to maintain standards and rules for IP
through the use of legislation and governmental regulatory authorities.  
\end{comment}


\newpage

\printbibliography[
heading=bibintoc,
title={References}
]


\begin{figure}[H]
\centering
\includegraphics[scale=0.18]{logo.png}
\caption{Sample Figure}
\label{fig:image1}
\end{figure}

Sample Reference\cite{einstein}


\begin{appendices}
\chapter{Some Appendix}
The contents...
\end{appendices}

\end{document}